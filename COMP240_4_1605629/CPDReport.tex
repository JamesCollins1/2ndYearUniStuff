% Please do not change the document class
\documentclass{scrartcl}

% Please do not change these packages
\usepackage[hidelinks]{hyperref}
\usepackage[none]{hyphenat}
\usepackage{setspace}
\doublespace

% You may add additional packages here
\usepackage{amsmath}

% Please include a clear, concise, and descriptive title
\title{CPD Report}

% Please do not change the subtitle
\subtitle{Semester 2 Report}

% Please put your student number in the author field
\author{1605629}

\begin{document}

\maketitle

\section{Introduction}
During this semester I have faced many new challenges, most of which I have been able to overcome. However, for those that I haven't I have put in place some methods with which I should be able to overcome them. They are as follows.

\section{Challenging Areas}

\subsection{Group game and Team Planning - Interpersonal Domain}
One major problem that my team this year suffered from was a lack of planning, both when it comes to planning individual work and planning out team member roles and areas to work on. That isn't to say that we didn't plan our work, or what we are working on, rather, we didn't spend enough time planning or plan thoroughly enough. This meant that there were many 'hiccups' throughout development when two team members might be working on the same thing independantly from one another, which slowed down development as we were effectively wasting time. Another issue was that we ended up having a lot of issues with the workflow between different disciplines. This could have been helped if we had someone that kept track of everyone's blockers, and new what time tasks should be completed by. This could cause an even larger problem for next year's team as we only have 2 programmers, so we need to be 100 percent sure about what work each other is doing.I have already been putting changes into effect with the new team for next year's development project. One of these major changes was to ensure we each knew what the other team member's specialisms were. This is a basic change, but it allows us to know who to go to when dealing with specific areas. It also allows us to see if there is any crossover between different team embers and better decide which areas they should focus on. While prototyping we have already noticed it having a positive effect. We have also assigned a team member to act as a 'Producer'. This means that team member takes extra time to ensure that no team member is being blocked by anything, and will aim to get other team members to remove any blockers to allow for the optimum work rate. We have set the first two sprints of next year, and the prototyping stage over summer as the trial period, whereby we will see if there is a marked improvement with the inclusion of a producer. If not, we will either try a different discipline as producer, such as a designer, or remove the role entirely dependant on results.

\subsection{Dealing with Stress during Hand-In Periods - Affective Domain}
Although this seems like a pretty general topic, the hand-ins towards the end of this semester have shown me that it is a top priority for me. During the hand-ins I have suffered from a very large amount of stress, leading to health issues such as insomnia. It took these health issues to realised that I really need to be able to manage and decrease the amount of stress I put on myself. I think the best way of acheiving this is to ensure that I am spending enough time to relax by spending time away from my PC. To do this I will aim to spend 1 hour away from the PC for every 3 that i am working. This allows me to take time to unwind during work. I feel like this is a good balance between working and resting. I will be trying this during the prototyping of next year's game and during the first semester of work in 3rd year for the deadline period. If this proves to be an effective method over this period then I will maintain this, otherwise I shall re-evaluate the times, increasing or descreasing the ratio accordingly. 

\subsection{Working on Production Modules throughout Development- Dispositional Domain}
I have found that I tend to work on production tasks, such as the comp250 Bot, or Comp 260 Server-Client task when I get them, but that tails off, and I find a lot more work needs to be done towards the end of the project. This leads to a 'crunch' period around deadlines which is both bad for the module work and my health. To fix this I have planned to spend at least 4 hours a week on each module. Although this seems quite self-evident, I believe that this will have a very large effect on the quality of my work. I will be doing this over the first semester of third year will show a marked improvement in the quality of the work, and actually might help me to deal with the stress too.

\subsection{Maintaining a level of Competency in multiple Languages - Cognitive Domain}
One major problem I have encountered this semester was the fact the my Csharp  knowledge had declined massively as I had not had to work in it for quite a while. As such I have decided that I need to start maintaining a certain level of competence within some varied programming languages. I have chosen to practice C++, CSharp, and Java because I feel they have been the most prominently used languages when I have checked job requirements. I have also found that we have been using these languages the most on the course. I will work on this by participating in the 'Code Wars' Website, a website that sets you daily challenges for your desired languages. I will do one challenge in each language per day to ensure I maintain my language skills. these challenges can range from simple tasks to complex algorithms. I will be doing this over the summer holiday, and during the first semester of 3rd year. I expect that doing this over summer will allow me to enter third year without having to spend any time re-aclimatising to the programming languages.
 
\subsection{Working on my UML skills - Procedural Domain}
Another specific issue I had this semester, that I would really like to improve upon was the use of UML. In previous CPD reports I have mentioned that I need to plan out my work and my code. I feel like the use of UML diagrams will help me plan in the most effective way. I also feel like it's a skill that I have forgotten from first year, but one that i would heavily use in industry. To improve my knowledge of UML I will be using many tutorials on the internet, such as the IBM developerWorks page on UML (https://www.ibm.com/developerworks/rational/library/769.html). I have alreadys started reading this and it has helped quite a lot when it comes to my general understanding of UML. I would be doing these over the summer holiday period, with the intention of  being able to use it during the game development project in 3rd year. If I have not been able to grasp this by the start of 3rd year then I will have to try and find some more formal learning material. Whether this is a payed tutorial set on the internet, or asking my tutors for further help remains to be decided, however, both would be effective.

\section{Conclusion}
There has definitely been an improvement in my level of skill since the first semester, with many of the plans I formed and followed from the first semester successfully removing many of the problems I encountered. however, the new year, larger game projectsand more complex coding tasks has led me to encounter and be negatively effected by many new problems. I believe some will be relatively easy to overcome, but others will take a bit longer to completely fix. However, I also believe that when I achieve the goals set above I will be able increase my work rate and improve my grade by a very large amount.
\end{document}